% !TEX root = ../report.tex

\section{Introduction}

The Internet of things (IoT) is a continuously growing network of commonplace
devices. They can be physical like wireless sensors and smart phones or virtual
like services. The IoT transforms how we interact with our environment. Today
almost any electronic device can be bought as a smart version like microwaves,
washing machines, light bulbs, door locks and many other "things". All these
belong to the IoT network. The IoT has several application domains. It can be
used personally or in enterprises \cite{Mahmoud2015}. The personal and social
application domain is for connecting people to their environment and to other
people. The industries and enterprises domain enables different activities in
and between organizations of for example the finance and banking sector. Also
IoT can be used in for example breeding or energy management where it is used
for service and utility monitoring. The last application domain is
transportation. Which encompasses smart cars and infrastructure like traffic
lights.

Ever since there were the first electronic systems to assist drivers in control
of their cars there were Electronic Control Units (ECUs) included in these
vehicles. ECUs consist of micro-controllers and sensors. Nowadays cars have not
only normal Driver Assistance Systems like antilock breaking systems. They also
have Advanced Driver Assistance Systems (ADAS) which help people with
safety-critical functionality like parking or emergency breaking. To realize all
these functions a multitude of in-vehicle networks (IVNs) is needed to connect
all the ECUs and the sensors. These IVNs were originally closed off and had no
connection to other networks. So they were designed with no security in mind
whatsoever. Recently, cars have been connected to a multitude of networks like
3G/4G mobile phone networks, Bluetooth and vehicular ad hoc networks (IEEE
802.11p). So the IVNs have become vulnerable to different types of attacks while
joining the IoT. The most used type of IVN today is still the Controller Area
Network (CAN). It was introduced in 1983 by the Robert Bosh GmbH. Because in
these days IVNs were closed of there were no security features included in the
CAN protocol like encryption or message authentication \cite{Avatefipour2017}.
This absence of inherent security features in the protocol led to successful
demonstrations of security breaches \cite{Koscher2010}. It was possible to
reprogram ECUs via physical and wireless connections. The car could even be
monitored and controlled remotely. Since then there have been many automotive
attack demonstrations \cite{Hoppe2011,Checkoway2011,Cheah2017}.

This report focuses on recent CAN authentication research. Two initially
proposed authentication protocols named LeiA and vatiCAN secure the CAN network
against adversaries that do not control code execution on ECUs in the network
\cite{Nurnberger2016,Radu2016}. Later these protocols were improved by VulCAN
which is an efficient approach to implement secure distributed automotive
control software on lightweight trusted computing platforms \cite{VanBulck2017}.
VulCAN adds another layer of security by relying on trusted hardware and a
minimal software Trusted Computing Base (TCB). Additionally it was shown that
VulCAN provides sufficient performance to be used in automotive real-time
applications.

In the following chapter \ref{sec:communication-networks} I will provide an
overview of different IVNs like CAN, Time-Triggered Networks, Low-Cost
Automotive Networks and more. Also I will introduce the general types of
security measures in IVNs. Afterwards chapters \ref{sec:leia} and
\ref{sec:vatican} will explain LeiA and vatiCAN and how each of them work in
general. Chapter \ref{sec:vulcan} will talk in depth about VulCAN. I will
explain how it achieves improvements over LeiA and vatiCAN using Sancus 2.0.
Finally there will be a discussion in chapter \ref{sec:discussion} connecting security in IVNs to security in
the general IoT sector and the world wide web.

% !TEX root = ../report.tex


\section{In-vehicle Communication Networks and Security}\label{sec:communication-networks}

The different functions of a vehicle have very different requirements in
performance and safety needs. Therefor the quality of service needed from the
communication system varies (e.g.\ response time or bandwidth). Normally there
are different functional domains which divide the in-car embedded
systems~\cite{Navet2017}. There are the safety-critical domains ``powertrain''
(e.g.\ engine control) and ``chassis'' (e.g.\ steering) that need a deterministic
real-time behavior. The functions of the ``body'' domain that controls for
example dashboard, wipers, lights and windows need to exchange many information
of small size between each other. Other domains like ``telematics'' and
``multimedia'' have for example increased requirements in bandwidth and
confidentiality.

A multitude of different networks resulted out of this diversity of
requirements. Therefor the Society of Automotive Engineers (SAE) created in 1994
a classification for automotive communication protocols. This classification is
based on data transmission speed and functionality. There were 4 different
classes defined that are labeled class A to class D~\cite{Ali2017}. Class A
networks have a speed lower than 10 kb/s. They are used for convenience features
such as trunk release or electric mirror adjustment. Examples for Class A
networks are LIN and TTP/A. Class B has a medium speed of 10 to 125~kb/s.
Networks of this classification are for general information between ECUs from
for example sensors. Main representatives of this class are J1850 and low-speed
CAN\@. High speed networks of class C have a speed between 125~kb/s and 1 Mb/s and
are used for real time control like the power train or vehicle dynamics.
High-speed CAN falls into this classification. Above the high speed
classification C there is class D\@. Every communication protocol faster than 1
Mb/s fall into this category. They are normally used either for multimedia
applications (e.g. MOST) or for hard real time critical functions like X-by-Wire
applications (e.g. TTP/C or FlexRay). Networks this fast, like FlexRay, can also
be used as gateways between sub-systems.

In modern vehicles it is normal that there are many networks of different types.
A BMW 7 series car from 2008 implements for example multiple LIN buses, a MOST
and a FlexRay bus and additionally four CAN buses~\cite{Kellermann2008}. All of
these networks are normally interconnected by gateways.

In-vehicle networks play a crucial role in keeping the embedded systems in a
safe state. Depending on the network it can be for example more or less
difficult to identify failed nodes. Also some networks can meet hard real-time
constraints and some cannot. In general there are two main paradigms in
automotive communication, event-triggered and time-triggered
communication~\cite{Navet2017}. If the communication is consisting of
asynchronous events, it follows the event-triggered paradigm. In such systems it
is crucial to avoid conflicts while sending events from multiple sources in
parallel. Event-triggered systems use the bandwidth effectively and are easy to
extend with new network nodes. If network communication is synchronous, so every
nodes sends at predefined time slot in a defined interval, then the
communication follows the time-triggered paradigm. In general accessing a medium
this way is called Time Division Multiple Access (TDMA). These systems are
perfectly predictable but require to be statically defined up front. If there
are new nodes introduced to the network the schedule has to be changed, so it is
more complicated to extend these systems. Also the bandwidth is not used very
efficiently and the response times are longer then in event-triggered systems.
However, missing messages can be identified rather easily. Because both paradigms
have up and down sides. Normally both types of communications are needed for
different features in an embedded system of a vehicle. Hence some networks like
FlexRay provide both types of communication alongside each other.

In the following sub-chapters I will describe the CAN network and give an
overview about some of the most representative other networks.

\subsection{CAN Network}

The CAN network was introduced in 1983 by the Robert Bosh GmbH~\cite{Navet2017}.
It uses a twisted pair of copper wires as a bus between the nodes. Depending on
the speed used, it is either classified and used as a SAE class B or class C
network. Aside the used speed there are two versions of CAN\@. Version 2.0A and
2.0B which differ in the length of the identifier used when sending data. CAN
uses a Non-Return-to-Zero bit representation with a bit stuffing of 5~bits.
Which means the bits will be represented by continuous levels of voltage. To be
able to stay in sync when sequences of the same bit value are sent over the wire
there will be the other bit value ``stuffed'' in after every 5 same consecutive
values. The receivers know this and can properly ``destuff'' the bit sequence.
So all CAN nodes can stay in sync using this bit stuffing. 

For CAN to be able to bound the respond times it uses a priority system. The
lower the identifier used to send data the higher the priority. To realize these
priorities the physical layer needs to implement an ``AND'' scheme. Only when
everyone simultaneously sends a 1 the resulting bus level is 1. If one node on
the bus sends a 0, the resulting bus level has to be a 0. So the 0 is the
dominant value on the bus and overrides a 1. With this mechanism it is easy to
realize the priority system.

The normally used version of CAN is 2.0A because it provides with \(2^{11}\)
possibilities enough identifiers so I will focus on its specs in the following.
Communication over the CAN bus is organized into frames of a maximum size of
135~bits if all overheads are included. A frame starts with a 1~bit Start Of
Frame. Following this start there is an 18~bit header. The header contains the
11~bit identifier, the Remote Transmission Request~(RTR) bit and the Data Length
Code~(DLC). The RTR distinguished between data frames and data request frames.
The DLC provides the length of the data following after the header. This data
can have at most 8~bytes. After the data follows a 15~bit Cyclic Redundancy
Check (CRC) for ensuring data integrity. After the CRC follows the
Acknowledgement field (Ack). With the Ack a sender is possible to know that a
sender has received the frame. However it is not possible to distinguish who
received it. At the end of the frame is the End Of Frame field followed by the
intermission bits. The intermission bits are the minimal number of bits after
which it is allowed to send a new data frame.

On an idle CAN bus every node can send a data frame at any time. If multiple
nodes want to send a frame simultaneously and collide, the priority based
arbitration decides which node is allowed to send data. This works with the help
of the physical layer implementing the ``AND'' scheme. So while sending
identifier and RTR bit the nodes are observing the bus level. If the node
observes a bit of its own to be overridden by a dominant bit (0 is dominant over
1) it stops sending data because another node with a smaller identifier plus RTR
is sending at the same moment which has priority. The node which stopped sending
needs to wait until the bus becomes idle again and then tries to send the data
frame again.

For the priority-based arbitration to work, the signal of a bit needs to
propagate to all other nodes and back before sending the next bit. So the
physical length of the bus restricts the speed with which data can be send. On a
40 meter bus a maximum speed of 1~MBit/s is possible while only 250~KBit/s can
be achieved over a 250 meter long bus~\cite{Navet2017}. This restriction has
lead to optimizations by the car manufacturers using ``traffic shaping''. One
example would be to use an offset for important periodic messages so they will
not interfere with each other~\cite{Navet2009}

The CAN protocol has different mechanisms to detect possible errors. One
mechanism would be to use the CRC to validate the data integrity. If an error is
detected by a node it will send six consecutive dominant bits which will cancel
the current data frame. This will alert all nodes on the bus that an error
occurred and a new arbitration phase can start. This delays the message sending
and possible deadlines may be missed this way. The possible start of a new frame
takes 17 to 31~bits after an error was detected. CAN also includes some
fault-confinement mechanisms for failures for example on the hardware level of
the micro-controller or communication controller. These mechanisms normally
involve the counting of failures and successful delivery of frames. However each
node is responsible itself to execute these mechanisms. So the relevance of
these mechanisms is questionable~\cite{Navet2017}. If there is an error for
example with the oscilloscope a node could be sending unknowingly many dominant
bits. This would be one manifestation of the ``babbling
idiot''~\cite{Pimentel2009}. More mechanisms are needed if the protocol is used
for safety-critical functions. There are lots of different solutions for
different problems but there is no formal verification for these mechanisms used
together~\cite{Navet2017}.

The CAN protocol only defines the physical layer and the Data Link layer but
there are higher level protocols like AUTOSAR which use the CAN protocol.

\subsection{Other Representative Networks}

There are other priority bus systems besides CAN like J1850 and VAN but they are
used less and less in favor of CAN~\cite{Navet2017}. In 2012 Robert Bosch GmbH
introduced a new CAN version named CAN FD~\cite{Hartwich2012}. It distinguishes
itself from standard CAN through higher data rates outside of the arbitration
phase and the possibility for larger data fields. Another advantage is the easy
migration from CAN to CAN FD because only the underlying communication layer has
to be changed. The nodes using CAN can remain unchanged.

Event-triggered networks such as CAN are not suited for all use cases of IVNs.
Event-triggered systems have lightweight protocols with dynamic arbitration per
message and good use of bandwidth. However they are hard to verify formally in
regard to e.g.\ bounded response times. If deterministic real time behavior is
needed for e.g.\ X-by-Wire Systems, it is more suitable to choose a
time-triggered network based on TDMA~(Time Division Multiple Access). They have
protocols with static arbitration and need a clock synchronization. The two TDMA
based networks suitable for serving as a gateway between IVNs or as a basis for
safety critical applications in vehicles are FlexRay and
TTEthernet~\cite{Navet2017}. 

The FlexRay specification was initially created by the FlexRay consortium which
was an alliance of car manufacturers, semiconductor and electronic systems
manufacturers (established in 2000). FlexRay is a very flexible network aiming
to provide advantages from the time-triggered and the event-driven world by
integrating both schemes into one protocol. It is possible to scale the ratio
between the time-triggered and the event-driven part of the protocol for
adapting the network to the use case. Theoretically it is possible to use just
one part of the protocol, the event-driven or time-triggered one.

Besides high speed and real time requirements, there are also parts of the
in-vehicle networks with very few requirements in comparison. Their primary
requirement is to have low costs. There are a multitude of such networks. The
Local Interconnect Network (LIN) and TTP/A are two representatives of these
networks whereupon TTP/A is not used in production vehicles and very similar to
LIN~\cite{Navet2017}, so I will not go into detail about TTP/A. Low-cost
automotive networks are normally used to control seats, doors, etc. They are not
only cheap because of the simple communication protocol but also because of the
low requirements in regard to micro-controller hardware. LIN is a master/slave
network with a single master node polling all the slave nodes for information.
The slave nodes are not allowed to initiate communication. Also LIN is able to
send nodes to sleep and to wake them up which is very important in the
automotive context. It works with just a single copper cable with a data rate up
to 20~KBit/s.

The third big IVN domain in vehicles is multimedia and infotainment networks.
The de-facto standard in this domain is the Media Oriented System Transport
(MOST)~\cite{Navet2017} developed by the MOST Corporation~\cite{MOST2018} which
is a consortium of car manufacturers and suppliers. MOST uses Polymer Optical
Fiber or coaxial transmission as the physical layer. It integrates the
possibility to transport standard ethernet frames. So it is possible to
transport content and information for example from the web. The current MOST
provides speed up to 150~MBit/s. However the MOST Corporation is working on a
new version with speed up to 5~GBit/s to be able to contend with automotive
ethernet which is an upcoming competitor.

Today, automotive ethernet is named as the future of in-vehicle networks.
However it has still not oust its competitors. Though in 2022 it is likely that
there are more ethernet ports inside of vehicles than anywhere
else~\cite{Ixia2014}. There are two major representatives of automotive ethernet
networks which are Ethernet Audio/Video Bridging (AVB) and TTEthernet. Ethernet
AVB works with over-provisioning and prioritization while the FlexRay competitor
TTEthernet is a time-triggered switched ethernet protocol utilizing TDMA\@.
TTEthernet provides mixed-criticality temporal requirements. It describes three
types of traffic streams with upper bounds on latency and jitter for two of the
three types. TTEthernet was designed for the most critical use-cases even
including aeronautic applications. Ethernet AVB was designed for low-latency
streaming services over ethernet by a consortium of car manufacturers and
suppliers called AVnu Alliance~\cite{AVNU2018}.

Automotive ethernet also needs some adjustments of the physical layer to work
properly in the vehicular environment in regard to cost, power saving modes,
robustness to environmental condition, etc. A physical ethernet layer developed
by the OPEN Alliance SIG~\cite{SIG2018} for the vehicular environment is
BroadR-Reach. It uses low cost unshielded copper wires. 

\subsection{Security Measures}

In-vehicle communication needs to be secured end-to-end so that no one else than
the intended receiver can read the messages, no one is able to modify message
content unnoticed and unauthorized parties are not able to participate in the
communication~\cite{Lemke2006}. There are three elementary security practices
for achieving security in IVN communication. This report focuses on the first
elementary practice: Controller Authentication.

\subsubsection{Controller Authentication}

Every valid controller sending messages in the IVN needs to be authenticated. So
that it is possible to distinguish messages from unauthorized parties and to be
able to process these separately or just discard them. A simple solution could
be the following. Every valid controller holds a signed certificate which
consists of a controller identifier, a public key and the authorizations for
this controller. The responsible gateway possesses a list of trusted Original
Equipment Manufacturers (OEMs) public keys. The certificates of the controllers
are signed by the respective private keys of the OEMs. This way the gateway can
decide which controller has which authorizations and only allow trusted OEM
controllers to participate.

\subsubsection{Encrypted Communication}

A fundamental part of secure IVN communication is the encryption of the network
buses. Because of the specific hardware constraints in the automotive domain
normally a mixture of asymmetric and symmetric encryption is used to meet the
requirements on security and performance. Asymmetric encryption could be used to
secure the acquisition of the symmetric keys. The symmetric encryption would
then be used to secure all bus communication.

\subsubsection{Gateway Firewalls}

Gateways should restrict unauthorized messages to be send into (high
safety-relevant) networks. This restriction can work per controller if there are
message authentication codes or digital signatures included in each message. If
it is technically not possible to include these into each message the only
possibility is to grant all controllers of the subnet specific authorizations.

\subsection{Automotive Open System Architecture and compliant CAN Authentication} 

AUTOSAR is a standard created by different vehicle manufacturers, suppliers,
service providers and companies from the automotive electronics, semiconductor
and software industry to overcome scalability and increasing development costs
for ECUs. It specifies a software framework to standardize the ECU software.
Since version 4.2 the specification includes security for CAN\@. It stipulates
128~bit keys and Message Authentication Codes (MACs) of a minimal length of
64~bit. They also specify the use of counters or timestamps to provide freshness of
the messages. The authentication mechanism needs to be \textit{backwards
compatible} to standard CAN to not destroy legacy ECUs that do not need
authentication because they are not safety critical.

There are already multiple published CAN authentication protocols. But most of
them fail to comply to the AUTOSAR specification. CANAuth~\cite{Herrewege2011}
and LiBrA-CAN~\cite{Groza2012} are protocols designed for CAN+ and not CAN
itself. They need new CAN transceivers and more powerful ECUs.
MaCAN~\cite{Bruni2014} is an authentication protocol designed for CAN and does
not need upgraded hardware. However, it splits the CAN payload into 4~byte data
and 4~byte MAC\@. This change in the payload field hinders backward
compatibility and does not conform to a minimal length of 8~byte for the MAC\@.
LCAP~\cite{Hazem2012} uses many CAN identifiers and only 2~byte MACs. 

To my best knowledge there are only three AUTOSAR-compliant protocols published
at the moment: LeiA~\cite{Radu2016}, vatiCAN~\cite{Nurnberger2016} and
VulCAN~\cite{VanBulck2017}. While LeiA and vatiCAN are pure software protocols
which provide security against the impersonation of protocol participants, the
VulCAN protocol enhances these two software protocols with additional
system-level security guarantees to provide security against directly
compromised ECUs.


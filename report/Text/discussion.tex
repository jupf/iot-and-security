% !TEX root = ../report.tex

\section{Discussion}
\label{sec:discussion}

The CAN network standard is now 25 years old. Although there are a lot of
published demonstations showing the risks that utilise unsecured
CAN~\cite{Koscher2010,Hoppe2011,Checkoway2011,Cheah2017}, no real change has
happened till today. The AUTOSAR standard included message authentication in
their standards already a few years ago but still, there are only three
published protocols that implement these standards. The software-only approaches
of VatiCAN and LeiA fulfill the most needed authentication protocol guarantees
while providing backwards compatibility to legacy CAN components. However, both
these protocols assume that from the attacker targeted ECUs are not compromised.
In presence of demonstrations showing a wide range of attacks on automotive
software and update mechanisms~\cite{Checkoway2011,Koscher2010} these
software-only approaches seem untrustworthy when it comes to the security of
vehicles which decides over life and death.

VulCAN took these software-only approaches and used Sancus to guarantee
additional system-level securities to close exactly this attack vector. VulCAN
is resistant to arbitrary code execution even on targeted ECUs. As long as the
used Sancus hardware is trusted (Sancus has a hardware-only TCB), VulCAN
provides a trustworthy approach to secure in-vehicle embedded systems. Of course
this comes at a higher price in comparison to software-only approaches but how
should these protect themselves against arbitrary code execution on an ECU\@?
They just do not provide enough security guarantees to seem trustworthy.
Additionally VulCANs possibility to shield legacy ECUs provides a good way to
transition in the direction of a secure in-vehicle network which would be impossible for a software-only approach.

The general Internet of Things sector is growing rapidly. There are up to 26
billion IoT devices expected in 2020~\cite{gartner13}. And according
to~\cite{Viega2012} the security of embedded devices in 2012 is a ``mess''.
Since then a few years passed and there is lots of research in this sector.
Sancus is exactly one of these published research works. Even if secure
communication and authentication is a well researched topic today, the IoT
domain has very specific requirements which prohibit the use of these well
defined and tested standards. IoT devices uses different network stacks and
often they have very few computing power. This makes it impossible to just use
standards like TLS~\cite{TODO}. Although the VulCAN protocol mechanisms are part
of the TLS standard in similar ways. TLS also uses symmetric cryptography to
calculate MACs and asymmetric crypto, like the AS, to authenticate participants.

However, the real problem is not designing secure embedded devices (although this is already hard), but making all these billion devices which are already sold secure. There are so many insecure IoT devices sold already that it is near impossible to secure all of them, even if it were wide-spread topic spoken about in the world and all the vendors putting much effort into it. Sadly this is not the case and closely resembles the vehicle domain. There are so many vehicles used in the world and with all the demonstrated security risks in mind, there is no real public conversation going on about the security of these vehicles. If we look at recent events that happened involving german car manufacturers (diesel scandal~\cite{spiegel17}), we can guess the willingness of the car manufacturers to make upgrades to already sold cars of any kind. In germany these car manufacturers did not decide to fix these cars they sold under wrong specifications. So why would they be willing to fix security issues in already sold cars nobody is talking about?

